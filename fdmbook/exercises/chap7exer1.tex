
\exercise[(Convergence of midpoint method)]{7.1}

Consider the midpoint method $\unp = \unm + 2kf(U^n)$ applied to the test
problem $u' = \lambda u$.  The method is zero-stable and second order
accurate, and hence convergent.  If $\lambda<0$ then the true solution 
is exponentially decaying.

On the other hand, for $\lambda<0$ and $k>0$ the point $z=k\lambda$ is never
in the region of absolute stability of this method (see Example 7.7),
and hence the numerical solution should be growing exponentially for any
nonzero time step.  (And yet it converges to a function that is exponentially
decaying.)

Suppose we take $U^0=\eta$, use Forward Euler to generate $U^1$, and then
use the midpoint method for $n=2,~3,~\ldots$.  Work out the exact solution
$U^n$ by solving the linear difference equation and explain how the apparent
paradox described above is resolved.

