
% Some latex examples showing how to typeset various things needed
% for writing about finite difference methods.

% Some macros from the file macros.tex are used.

\documentclass{article}
\usepackage{amsmath}
\usepackage{epsfig,psfrag}

\setlength{\textwidth}{6.5in}
\setlength{\textheight}{8.9in}
\setlength{\voffset}{-1in}
\setlength{\oddsidemargin}{0in}
\setlength{\evensidemargin}{0in}


% a few handy macros

\newcommand\matlab{{\sc matlab}}
\newcommand{\goto}{\rightarrow}
\newcommand{\bigo}{{\mathcal O}}
\newcommand{\half}{\frac{1}{2}}
%\newcommand\implies{\quad\Longrightarrow\quad}
\newcommand\reals{{{\rm l} \kern -.15em {\rm R} }}
\newcommand\complex{{\raisebox{.043ex}{\rule{0.07em}{1.56ex}} \hskip -.35em {\rm C}}}


% macros for matrices/vectors:

% matrix environment for vectors or matrices where elements are centered
\newenvironment{mat}{\left[\begin{array}{ccccccccccccccc}}{\end{array}\right]}
\newcommand\bcm{\begin{mat}}
\newcommand\ecm{\end{mat}}

% matrix environment for vectors or matrices where elements are right justifvied
\newenvironment{rmat}{\left[\begin{array}{rrrrrrrrrrrrr}}{\end{array}\right]}
\newcommand\brm{\begin{rmat}}
\newcommand\erm{\end{rmat}}

% for left brace and a set of choices
\newenvironment{choices}{\left\{ \begin{array}{ll}}{\end{array}\right.}
\newcommand\when{&\text{if~}}
\newcommand\otherwise{&\text{otherwise}}
% sample usage:
%  \delta_{ij} = \begin{choices} 1 \when i=j, \\ 0 \otherwise \end{choices}


% for labeling and referencing equations:
\newcommand{\eql}{\begin{equation}\label}
\newcommand{\eqn}[1]{(\ref{#1})}
% can then do
%  \eql{eqnlabel}
%  ...
%  \end{equation}
% and refer to it as equation \eqn{eqnlabel}.  


% some useful macros for finite difference methods:
\newcommand\unp{U^{n+1}}
\newcommand\unm{U^{n-1}}

% for chemical reactions:
\newcommand{\react}[1]{\stackrel{K_{#1}}{\rightarrow}}
\newcommand{\reactb}[2]{\stackrel{K_{#1}}{~\stackrel{\rightleftharpoons}
   {\scriptstyle K_{#2}}}~}

  % define some macros that are generally useful

% some more macros, useful for this document
\newcommand{\sixth}{\frac{1}{6}}  
\newcommand{\bx}{\bar x}

%-------------------------------------------------------------------
\begin{document}          

\hfill\vbox{\hbox{AMath 586}\hbox{Latex samples}\hbox{Spring, 2007}}

\vskip .3in

%-----------------------------------------------------------------

\vskip 10pt
\noindent 
{\Large \bf Equations:}
\vskip 5pt
In-line equations or math symbols have dollar signs around them, e.g., 
the function $u(t)$ satisfies $u' = \lambda u$.

Displayed equations without equation numbers can be made either with
\[
\int_0^\infty f(x) \, dx
\]
or with 
\begin{equation*}
\|A^{-1}\|_\infty = k\sum_{m=1}^N |(1 + k\lambda)^{N-m}|
\end{equation*} 
To add an equation number, leave off the * in the equation environment,
\begin{equation} \label{sum2}
\sum_{j=1}^r a_{ij} = c_i,  \quad i=1,~2,~\ldots,~r.
\end{equation}
The label command lets you refer to this equation later as equation
(\ref{sum2}), or using the shorthand macro, \eqn{sum2}.

Labels can be added to other numbered things like figure captions too.


%-----------------------------------------------------------------

\vskip 10pt
\noindent 
{\Large \bf Equations lined up:}
\vskip 5pt

\begin{equation*}
\begin{split}
u''(t) &= f_u(u(t),t)u'(t) + f_t(u(t),t)\\
&=  f_u(u(t),t)f(u(t),t) + f_t(u(t),t)
\end{split}
\end{equation*} 

\noindent
Another example:

\begin{equation*}
\begin{split} 
Y_1 &= U^n\\
Y_2 &= U^n + \half k f(Y_1,t_n)\\
Y_3 &= U^n + \half k f(Y_2,t_n + k/2)\\
Y_4 &= U^n +  k f(Y_3,t_n + k/2)\\
\unp &= U^n + {k\over 6} [f(Y_1,t_n) + 2f(Y_2,t_n + k/2) \\
&\qquad\null + 2f(Y_3,t_n + k/2) + f(Y_4,t_n+k)]
\end{split}
\end{equation*} 

%-----------------------------------------------------------------

\vskip 10pt
\noindent
{\Large \bf Vectors and matrices:}
\vskip 5pt
Here's a vector $u \in \reals^3$:
\[
u = \bcm u_1 \\ u_2 \\ u_3 \ecm.
\]

A linear system:
\[
\bcm a_{11} & a_{12} \\  a_{21} & a_{22} \ecm
\bcm x_1 \\ x_2 \ecm = \bcm b_1 \\ b_2 \ecm.
\]

A tridiagonal matrix:
\[
A =\frac{1}{h^2} \bcm  h^2&0\\1&-2&1\\  &1&-2&1\\
&&\ddots&\ddots&\ddots\\ &&&1&-2&1\\ &&&&1&-2&1\\ &&&&&0&h^2 \ecm.
\]


%-----------------------------------------------------------------

\vskip 10pt
\noindent
{\Large \bf Butcher tableau (p. 131):}
\vskip 5pt

\begin{center}
\begin{tabular}{c|ccc}
$c_1$ & $a_{11}$ & $\ldots$ & $a_{1r}$\\
$\vdots$ & $\vdots$ & & $\vdots$ \\
$c_r$ & $a_{r1}$ & $\ldots$ & $a_{rr}$\\
\hline\\
&$b_1$ & $\ldots$ & $b_r$
\end{tabular}
\end{center}

\vskip 10pt

\begin{center}
\begin{tabular}{c|cccc}
0\\
1/2&1/2\\
1/2&0&1/2\\
1&0&0&1\\
\hline\\
&1/6&1/3&1/3&1/6
\end{tabular}
\end{center}

%-----------------------------------------------------------------


\end{document}
